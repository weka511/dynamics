\documentclass[]{article}
\usepackage{caption}
\usepackage{subcaption}
\usepackage{float}
\usepackage{url}
\usepackage{amsmath}
\usepackage{amssymb}
\usepackage{graphicx}
\graphicspath{{figs/}} 
%opening
\title{}
\author{}

\begin{document}

\maketitle

\begin{abstract}

\end{abstract}

\section{Introduction}



\section{My Problem Defined}

\section{Implementation of My Method}

\section{Results}
\begin{figure}[H]
	\includegraphics[width=\textwidth]{Lorentz.png}
\end{figure}
\section{Discussion}

\section{Plan}

\begin{enumerate}
	\item Dynamics
	\begin{enumerate}
		\item construct a symbolic dynamics
		\item count prime cycles
		\item prune inadmissible itineraries, construct transition graphs if appropriate
		\item implement a numerical simulator for your problem
		\item compute a set of the shortest periodic orbits
		\item compute cycle stabilities
	\end{enumerate}
	\item Averaging, numerical
	\begin{enumerate}
		\item  estimate by numerical simulation some observable quantity, like the
		escape rate,
		\item or check the flow conservation, compute something like the Lyapunov
		exponent
	\end{enumerate}
	\item Averaging, periodic orbits
	\begin{enumerate}
		\item  implement the appropriate cycle expansions
		\item   check flow conservation as function of cycle length truncation, if the
		system is closed
		\item   implement desymmetrization, factorization of zeta functions, if dynamics possesses a discrete symmetry
		\item  compute a quantity like the escape rate as a leading zero of a spectral
		determinant or a dynamical zeta function.
		\item  or evaluate a sequence of truncated cycle expansions for averages,
		such as the Lyapunov exponent or/and diffusion coefficients
		\item  compute a physically interesting quantity, such as the conductance
		\item  compute some number of the classical and/or quantum eigenvalues, if
		appropriate
	\end{enumerate}
\end{enumerate}

\end{document}
