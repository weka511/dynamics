% MyNamePHYS7123.tex

% add editing date at the top  whenever any file.tex is modified
% PC 							Nov 11:
% YL							Nov  1:
% http://publish.aps.org/esubs/revtextips.html

% this style for submission, web version:
\documentclass[pre,twocolumn,groupedaddress,showpacs,showkeys]{revtex4}

% this style while editing:
%\documentclass[pre,preprint,groupedaddress,showpacs,showkeys]{revtex4}

\input defs	  		% all definitions in defs.tex
				\begin{document}
				\title{
Title of My First Physical Review Article
				}\author{
My Name				}\email{gt??@prism.gatech.edu}
\affiliation{
		School of Physics\\
		Jawja Institute of Technology, Atlanta, GA 30332-0430, U.S.A}
				\date{\today} % or edit manually:
				%\date{November 11,:}

				\begin{abstract}
A novel methods for determining unstable periodic orbits of 
My Dynamical System ...............
is proposed and implemented. 

\end{abstract}
\pacs{95.10.Fh, 02.70.Bf, 47.52.+j, 05.45.+a}
\keywords{
periodic orbits, 
chaos, turbulence, ??? 
	} 
					\maketitle

\noindent
{\bf Georgia Tech PHYS 7224:}\\
\underline{\bf CHAOS, AND WHAT TO DO ABOUT IT }\\
{\bf course project, spring semester 2005}

\section{Introduction}

   My project is an investigation of a dynamical system
   that I find interesting - not only is it relevant to
   my thesis, but it 
   a deterministic system (PDE reduced to ODEs reduced to a map,...) and
   it maybe exhibits chaotic behavior. I work and work through
   a deadly cute but ubscure paper~\cite{einstein}.

The periodic orbit theory of classical chaos
is blah-blah of chaotic dynamical systems. 
The theory expresses all long time averages over chaotic
dynamics in terms of cycle expansions~\cite{DasBuch,PG97}.
sums over
periodic orbits  (cycles) 
ordered hierarchically according to the orbit length, stability, or action. 
If the symbolic dynamics is known, and the flow is
hyperbolic, longer cycles are shadowed by the shorter ones, and
cycle expansions converge  exponentially or even 
super-exponentially with the cycle length~\cite{Rugh92}. 


    In \refsect{sec:der} I derive my differential equation
which governs the evolution of blah-blah.
Simplifications due to symmetries and details of my
numerical implementation of the method are discussed in \refsect{sec:eqs}. 
In \refsect{sec:triumph} I test
the method on My System.
My results are summarized
discuss possible improvements of the method in \refsect{sec:sum}.
Why I failed to .... is explained in \refappe{appe:plan}.

\section{My Problem Defined}
\label{sec:der}

	Description of the problem, physical motivation

    
A periodic orbit is a
solution $(x,\period{})$, $x \in \reals^{d}$,
$\period{} \in \reals$ of the {\em periodic orbit condition}
\beq
f^{\period{}}(x) = x
\,,\qquad \period{} > 0
\label{e:periodic}
\eeq
for a given flow or discrete time mapping $x \mapsto f^t({x})$.
Our goal is to determine periodic 
orbits of flows defined by first order ODEs
\beq
\frac{dx}{dt}=v(x)
	\,,\qquad
 x \in \pS \subset \mathbb{R}^d
	\,,\qquad 
 (x,v) \in \bf{T}\pS  
\label{fl}
\eeq
in $d$ dimensions. Here $\pS$ is the phase space 
(or state space) in which evolution takes place, 
and the vector field $v(x)$ is smooth (sufficiently 
differentiable) almost everywhere.


\subsection{My Equations}
\label{sec:eqs}

	Basic equations .... I often refere to
\refeq{e:periodic}
and my \reffig{f:Poinc} is still missing the punchline.
But my results illustrated in 
\reffig{f:ks1}
give us hope.

Denote by $\delta_n$ the deviation of a point $x_n$ on the 
periodic orbit $p$ from the nearby point $y_n$, 
%$x(t_n)$
%could be written as $\{
\[
x_n  =y_n+\delta_n
	% \,,\qquad
	% {n=1,\ldots,N}
\,.
\] 
The deviations $\delta $ are assumed small, vanishing as
$L$ approaches $p$.
% represent  of the loop $L$ from the true periodic orbit  $p$. 

Let 
$ \pSpace(t) = \flow{t}{\pSpace} $
be the state of the system at time $t$ obtained by
integrating \refeq{fl}, and 
$\jMps(\pSpace,t) = d \pSpace(t)/d \pSpace(0)$ be
the corresponding {} obtained by integrating
\beq
\frac{d\jMps}{dt} = \Mvar \jMps
\,,\quad
\Mvar_{ij}=\frac{\partial v_i}{\partial x_j}
\,,\qquad
\mbox{with } \jMps(x,0)=\mathds{1} 
\,. 
\label{doa}
\eeq


\subsection{Marginally amusing tangent}
\label{sect:marg}

Numerically, two perils lurk in a direct implementation of the {} 
\refeq{doa}: 

(1) Ho hum

(2) Hee hee
 
 
\section{Implementation of My Method}
\label{sec:nm}

    As My System satisfies the  periodic boundary condition:

 
it is natural to ...

\subsection{Numerical implementation}

%
%%%%%%%%%%%%%%%%%%%%%%%%%%%%%%%%%%%%%%%%%%%%%%%%%%%%%%%%
\begin{figure}[t!]
	\includegraphics[width=2.8in]{figs/HMV.eps}
\caption{
 I am fetching my master's Poincar\'e section ret cheer.
        }
\label{f:Poinc}
\end{figure}  
%%%%%%%%%%%%%%%%%%%%%%%%%%%%%%%%%%%%%%%%%%%%%%%%%%%%%%%%%%%%%

    In a discretization of a loop, numerical
stability requires accurate  
discretization of  loop derivatives such as
 \[
v_n \equiv
\left.\frac{\partial }{\partial s}\right|_{ = (s_n)}
	\approx (\hat{D})_n
\,.
\]
In our numerical work we use the four-point approximation,
{\small 
\[ 
\hat{D} = 
\frac{1}{12h}
\!
\left( \begin{array}{ccccccccccc} 0&8&-1&&&\qquad
&&&&1&-8 \\ -8&0&8&-1&&\qquad &&&&&1 \\ 1&-8&0&8&-1&\qquad &&&&&\\
&&&&&\cdots&&&&&\\
&&&&&\qquad &1&-8&0&8&-1 \\
-1&&&&&\qquad &&1&-8&0&8 \\
8&-1&&&&\qquad &&&1&-8&0 
\end{array}\right)
\]
}%end {\small  
where $h={2 \pi}/{N}$. 
Here, each entry represents a $[d\! \times\! d ]$ matrix, $8 \to 8\mathds{1}$, 
{\em etc.}, 
with blank spaces are filled with zeros.  

The discretized version
with a fictitious time Euler step $\delta \tau$ is
\beq
\MatrixII
 {\hat{\Mvar}}{\hat{v}}
 {\hat{a}}{0} 
  \VectorII {\delta \hat{x}}{\delta \lambda} 
=\delta \tau \VectorII{\lambda \hat{v}-\hat{v}} {0}
\,. 
\label{mform} 
\eeq

    In my numerical work,
this matrix is inverted using the banded LU decomposition 
on the embedded
band-diagonal matrix.

\subsection{Initialization of the search}
\label{subsec:init}

    As is often the case, an understanding of the problem is 
a prerequisite to successful solution searches. We start by numerical 
integration of the dynamical system \refeq{fl}. Numerical experiments reveal
regions where a trajectory 

\subsection{Desymmetrization}
\label{subsec:desymm}

    My system is 
symmetric under ....

  
\section{The Hour of Triumph}
\label{sec:triumph}

    We have checked that the iteration of \refeq{mform} yields quickly and robustly
the short unstable cycles for My System.

\subsection{Periodic orbits of My System}

   The Kuramoto-Sivashinsky equation arises as an amplitude equation
for interfacial instability in a variety of contexts\rf{KurSiv}. 
In 1-dimensional space, it reads
\begin{equation}
u_t=(u^2)_x-u_{xx}-\nu u_{xxxx}, \label{kseq}
\end{equation}
 where $\nu$ is a ``super-viscosity'' parameter which controls the rate of
dissipation and $(u^2)_x$ is the nonlinear convection term. 
As $\nu$ decreases, the system undergoes a series of bifurcations,
leading to increasingly turbulent, spatio-temporally chaotic dynamics.

 If we impose the periodic boundary condition
$u(t,x+2\pi)=u(t,x)$ and choose to study only the odd solutions
$u(-x,t)=-u(x,t)$, $u(x,t)$ can be expanded in a discrete spatial
Fourier series,
\begin{equation}
u(x,t)=i\sum_{k=-\infty}^{\infty} a_k(t) e^{ikx}
\,,
\label{expan}
\end{equation} 
where $a_{-k}=-a_k \in \mathbb{R}\,$. In terms of the Fourier components, 
PDE \refeq{kseq} becomes an infinite ladder of ODEs,
\begin{equation}
\dot{a_k}=(k^2-\nu k^4)a_k-k\sum_{m=-\infty}^{\infty}a_m a_{k-m} \,. \label{ksf}
\end{equation}
 In numerical simulations we work with the Galerkin truncations of the
Fourier series since in the neighborhood of the strange attractor the 
magnitude of $a_k$ decreases very fast with
$k$,  high frequency modes playing a negligible role
in the asymptotic dynamics. In this way Galerkin truncations reduce the
dynamics to a finite but large number of ODEs. We work with $d=32$ dimensions 
in our numerical 
calculations. 
%
%%%%%%%%%%%%%%%%%%%%%%%%%%%%%%%%%%%%%%%%%%%%%%%%%%%%%%%%
\begin{figure}[t!]
	(a)~\includegraphics[width=1.8in]{figs/ks21.eps}%
	\hspace{0.2cm}%
	(b)~\includegraphics[width=1.8in]{figs/ks22.eps}
\caption{
 My System in the
 chaotic regime (greasy parameter
 $\nu=0.015$, $d=32$ modes truncation). 
 (a) An initial guess $L_1$, and 
 (b) the periodic orbit $p_1$ of period $\period{1}=0.744892$ 
 determined by My Method.
	}
\label{f:ks1}
\end{figure}  
%%%%%%%%%%%%%%%%%%%%%%%%%%%%%%%%%%%%%%%%%%%%%%%%%%%%%%%%%%%%%

\section{Discussion}
\label{sec:sum}

	Ultimate goals of the project was ...

I attained the  minimal goal of ...
	   
It was realistic to expect that ...
	   
Then I got incredibly lucky, and

In order to cope with the difficulty of finding periodic orbits in
high-dimensional chaotic flows, we have

My main result is 
     
My method uses information from ... .
The method is quite robust in practice.

\begin{acknowledgments}

I would like to thank Mason Porter for 
numerous helpful suggestions, and
my mother for bringing \refrefs{HLB96,CCP96} 
to my attention.

\end{acknowledgments}         

%%%%%%%%%%%%%%%%%%%%%%%%%%%%%%%%%%%%%%%%%%%%%%%%%%%%%%%%%%%%%
\appendix

\section{Project plan}
\label{appe:plan}

I intend to comment this appendix out when the project is finished
- for now it helps Predrag keep track of how far am I along the
plan.

Schedule of which part I  intend to deliver by which date:

\begin{enumerate}
	   \item{\bf Tue Mar 8:}
Will work out and integrate the equations for  ...
	   \item{\bf  Tue Mar 15:}
Will construct the Poincar\'e section
	   \item{\bf  Tue Mar 29:}
Will determine approximate symbolic dynamics for ...
	   \item{\bf  Tue Apr  5:}
Will find periodic orbits of ...
	   \item{\bf  Tue Apr 12:}
Will use cycle expansions to compute the average of ...
	   \item{\bf  Tue Apr 19:}
Will polish the project to high shine ...
	   \item{\bf  Tue Apr 26:}
Fix the last few quirks ...
	   \item{\bf  Tue May 2:}
Project deadline
\end{enumerate}


%%%%%%%%%%%%%%%%%%%%%%%%%%%%%%%%%%%%%%%%%%%%%%%%%%%%%%%%%%%%%

\input refs

\end{document}
