\documentclass[]{article}
\usepackage{caption,subcaption,graphicx,float,url,amsmath,amssymb,amsthm,tocloft,cancel,thmtools,braket}
\newtheorem{ex}{Exercise}
\newcommand\numberthis{\addtocounter{equation}{1}\tag{\theequation}}
\newtheorem{thm}{Theorem}
\newtheorem{defn}[thm]{Definition}
\newtheorem{cor}[thm]{Corollary}
\newtheorem{lemma}[thm]{Lemma}
%opening
\title{Worked problems from Chaos Book}
\author{Simon Crase}

\begin{document}

\maketitle

\begin{abstract}
Exercises and worked examples  from \cite{ChaosBook}.
\end{abstract}

\tableofcontents

\section{Lorenz Equations}

\subsection{Equilibrium Points}
From \cite{lorenz1963deterministic}
\begin{align*}
	\dot{x} =& \sigma(y-x) \numberthis \label{eq:lorenz:1}\\
	\dot{y} =& \rho x -y - xz \numberthis \label{eq:lorenz:2}\\
	\dot{z} =& xy-bz \numberthis \label{eq:lorenz:3}
\end{align*}
The equilibria are given by
\begin{align*}
	\sigma(y-x) =& 0\numberthis \label{eq:lorenz:eq:1}\\
	\rho x -y - xz  =& 0\numberthis \label{eq:lorenz:eq:2}\\
	xy-bz =& 0\numberthis \label{eq:lorenz:eq:3}
\end{align*}
Assuming $\rho, \sigma, b \ne 0$
\begin{align*}
	y =& x\\
	z =& \frac{xy}{b}
\end{align*}
One solution is $(x,y,z)= (0,0,0)$. The other solutions are given by substituting in \eqref{eq:lorenz:eq:2}:
\begin{align*}
	\rho x - x-x \frac{x^2}{b} =& 0\\
	\rho -1 - \frac {x^2}{b} =& 0\\
	x^2 =& b(\rho-1)\\
	x =& \pm \sqrt{ b(\rho-1)}\\
	(x,y,z)=& (\pm \sqrt{ b(\rho-1)},\pm \sqrt{ b(\rho-1)},\rho-1) \numberthis \label{eq:lorenz:eq:eq}
\end{align*}

\subsection{Exercise 11.4}
We convert  \eqref{eq:lorenz:1}, \eqref{eq:lorenz:2}, and  \eqref{eq:lorenz:2} to polar coordinates.

\begin{align*}
	x =& r \cos \theta\\
	y =& r \sin \theta \text{, we have}\\
	\dot{x} =& \dot{r} \cos \theta - r \sin\theta \; \dot{\theta}\\
	\dot{y} =& \dot{r} \sin \theta + r \cos\theta \; \dot{\theta}\\
	x\dot{x}+ y\dot{y} =& r  \dot{r} \cos^2 \theta -  \cancel{r^2 \cos \theta \sin\theta \; \dot{\theta}}+  r \dot{r} \sin^2 \theta + \cancel{ r^2 \cos\theta \sin \theta \; \dot{\theta}}\\
	=& r \dot{r} \numberthis \label{eq:rdot:1}\\
	x\dot{y}-y\dot{x}=& \cancel{r \cos \theta \dot{r} \sin \theta} +r^2 \cos^2 \theta  \; \dot{\theta} -\cancel{r \sin \theta \dot{r} \cos \theta} + r^2 \sin^2 \theta  \; \dot{\theta}\\
	=& r^2 \dot{\theta} \numberthis \label{eq:thetadot:1}
\end{align*}
Substituting  \eqref{eq:lorenz:1} and \eqref{eq:lorenz:2} in \eqref{eq:rdot:1}
\begin{align*}
	\dot r =& \frac{1}{r}\big[\sigma r^2 \cos \theta (\sin\theta-cos\theta) +  r^2 \sin \theta \big(\rho \cos\theta -sin\theta - \cos \theta z\big)\big]\\
	=& \sigma r \cos \theta \sin\theta -\sigma r \cos^2 \theta + r (\rho-z) \cos\theta \sin \theta-r \sin^2 \theta \\
	=& \big(\sigma +\rho -z \big)r \underbrace{\cos \theta \sin\theta}_\text{$\frac{\sin 2\theta}{2}$} -\sigma r \underbrace{\cos^2 \theta}_\text{$\frac{1+\cos 2\theta}{2}$}-r \underbrace{\sin^2 \theta}_\text{$\frac{1-\cos 2\theta}{2}$}\\
	=&  \big(\sigma +\rho -z \big)r \frac{\sin 2\theta}{2} - \sigma r \frac{1+\cos 2\theta}{2} -r \frac{1-\cos 2\theta}{2}\\
	=& \frac{r}{2}\big[\big(\sigma +\rho -z \big)\sin 2\theta - 1 -\sigma + \big(1-\sigma\big) \cos 2\theta\big] \numberthis \label{eq:lorenz:polar:r}
\end{align*}
Substituting  \eqref{eq:lorenz:1} and \eqref{eq:lorenz:2} in \eqref{eq:thetadot:1}
\begin{align*}
	\dot{\theta} =& \frac{1}{r^2}\big[ r \cos \theta\big(\rho r \cos \theta -r \sin \theta - r \cos \theta z\big) - r \sigma \sin\theta \big(r \sin \theta-r \cos \theta\big) \big]\\
	=& \cos \theta\big(\rho  \cos \theta - \sin \theta - \cos \theta z\big) -  \sigma \sin\theta \big( \sin \theta- \cos \theta\big)\\
	=& \big(\rho-z  \big) \cos^2 \theta - \sigma \sin^2 \theta + \big( \sigma-1 \big) \cos \theta \sin \theta\\
	=& \frac{\rho-z }{2}\big[1 + \cos 2 \theta\big] - \frac{\sigma}{2}\big[1 - \cos 2 \theta\big]+\frac{\sigma-1}{2} \sin 2 \theta\\
	=& \frac{1}{2}\big[\rho-z -\sigma + \big(\rho-z +\sigma\big)\cos 2 \theta + \big(\sigma-1\big) \sin 2 \theta\big] \numberthis \label{eq:lorenz:polar:theta}
\end{align*}
Converting \eqref{eq:lorenz:3} to polar coordinates gives
\begin{align*}
	\dot{z} =& r^2 cos\theta \sin\theta - b z\\
	=& \frac{1}{2} r^2 \sin 2\theta -bz \numberthis \label{eq:lorenz:polar:z}
\end{align*}
Note that the transformation $\theta \rightarrow \theta + \pi$ leaves $\cos 2 \theta$ and $\sin 2\theta$ unchanged, so \eqref{eq:lorenz:polar:r}, \eqref{eq:lorenz:polar:theta}, and \eqref{eq:lorenz:polar:z} are invariant.

\subsection{Pseudo-Lorenz}

The transformation $\theta \rightarrow \theta + \pi$ can also be written $x \rightarrow -x, y \rightarrow -y$. Following \cite{miranda1993proto}, we define new variables in \eqref{eq:lorenz:1},  \eqref{eq:lorenz:2}, and \eqref{eq:lorenz:3}, which are clearly invariant under $x \rightarrow -x, y \rightarrow -y$.
\begin{align*}
	u =& x^2-y^2 \numberthis \label{eq:miranda:stone:1}\\
	v =& 2 x y\numberthis \label{eq:miranda:stone:2}
\end{align*}
We also define an auxiliary variable:
\begin{align*}
	N =& \sqrt{u^2+v^2}\\
	=& \sqrt{(x^2-y^2)^2 + 4x^2y^2}\\
	=& \sqrt{x^4 - 2 x^2 y^2 + y^4  + 4x^2y^2}\\
	=& \sqrt{x^4 + 2 x^2 y^2 + y^4}\\
	=& \sqrt{(x^2+y^2)^2}\\
	=& x^2+y^2 \numberthis \label{eq:miranda:stone:N}
\end{align*}
The following identities will be useful.
\begin{align*}
	N+u= & x^2+y^2 + x^2 - y^2 \text{, from \eqref{eq:miranda:stone:1} and \eqref{eq:miranda:stone:N}} \\
	=& 2 x^2 \text{, whence}\\
	x^2 =& \frac{N+u}{2}\numberthis \label{eq:miranda:stone:x}\\
	N - u =& x^2+y^2 - x^2 + y^2 \text{, from\eqref{eq:miranda:stone:2} and \eqref{eq:miranda:stone:N}}\\
	=& 2 y^2 \text{, whence}\\
	y^2 =& \frac{N-u}{2}\numberthis \label{eq:miranda:stone:y}	
\end{align*}

We transform the Lorenz equation to the new coordinates.
\begin{align*}
	\dot{u} =& 2 x \dot{x} - 2 y \dot{y} \text{, from \eqref{eq:miranda:stone:1}}\\
	=& 2x\sigma(y-x)-2y(\rho x - y -xz) \text{, from \eqref{eq:lorenz:1} and \eqref{eq:lorenz:2}}\\
	=&( \sigma - \rho)(2xy) -2\sigma x^2  + 2y^2 + (2xy)z\\
	=& ( \sigma - \rho) v -\sigma (N+u) + (N-u) + vz \text{, using \eqref{eq:miranda:stone:2}, \eqref{eq:miranda:stone:x} and \eqref{eq:miranda:stone:y}}\\ 
	=& ( \sigma - \rho) v + (1-\sigma)N - (1+\sigma)u +vz \numberthis \label{eq:miranda:stone:de:u}\\
	\dot{v} =& 2 \dot{x} y + 2 x \dot{y} \text{, from \eqref{eq:miranda:stone:2}}\\
	=& 2\sigma(y-x)y+2x(\rho x-y-xz)  \text{, from \eqref{eq:lorenz:1} and \eqref{eq:lorenz:2}}\\
	=& 2 \sigma y^2 -2 \sigma x y + 2\rho x^2 -2xy -2 x^2 z\\
	=&  \sigma (N-u) - \sigma v + \rho (N+u) -v - (N+u) z \text{, using \eqref{eq:miranda:stone:2}, \eqref{eq:miranda:stone:x} and \eqref{eq:miranda:stone:y}}\\
	=& (\rho-\sigma) u - (\sigma+1)v + (\rho+\sigma)N - (N+u) z \numberthis \label{eq:miranda:stone:de:v}\\
	\dot{z} =& \frac{1}{2} 2xy - bz \text{, From \eqref{eq:lorenz:3}} \\
	=& \frac{1}{2}v - z \text{, from \eqref{eq:miranda:stone:2}} \numberthis \label{eq:miranda:stone:de:z}
\end{align*}
\section{Homework 2}

\subsection{Q1.3 Floquet Multipliers}

From \cite[Q1.3]{ChaosBook}
\begin{align*}
	\dot{q} =& p + q\big(1-q^2-p^2\big) \numberthis \label{eq:dot_q}\\
	\dot{p} =&-q + p\big(1-q^2-p^2\big) \numberthis \label{eq:dot_p}
\end{align*}
Transform to polar coordinates
\begin{align*}
	q =& r \cos{\theta}\\
	p =& r \sin{\theta}\\
	\dot{q} =& \dot{r} \cos{\theta}  - r \sin{\theta} \, \dot{\theta} \numberthis \label{eq:dot_q_polar}\\
	\dot{p} =& \dot{r} \sin{\theta}  + r \cos{\theta} \, \dot{\theta} \numberthis \label{eq:dot_p_polar}
\end{align*}
Substituting \eqref{eq:dot_q} and \eqref{eq:dot_p} in \eqref{eq:dot_q_polar} and \eqref{eq:dot_p_polar}
\begin{align*}
	\cos{\theta} \dot{q} + \sin{\theta} \dot{p} =& \dot{r}\cos^2\theta -\cancel{r \cos{\theta} \sin{\theta}\; \dot{\theta}} + \dot{r}\sin^2\theta+ \cancel{r \cos{\theta} \sin{\theta}\; \dot{\theta}}\\
	=& \dot{r}  \numberthis \label{eq:dot_r0}\\
	\sin{\theta} \dot{q} - \cos{\theta}\dot{p} =& \sin{\theta} \big[\cancel{\dot{r} \cos{\theta}}  - r \sin{\theta} \, \dot{\theta}\big] - \cos{\theta}\big[\cancel{\dot{r} \sin{\theta}}  + r \cos{\theta} \, \dot{\theta}\big]\\
	=& - r \big(\cos^2\theta+\sin^2\theta\big) \dot{\theta}\\
	=&-r \dot{\theta} \numberthis \label{eq:dot_theta}
\end{align*}
\begin{figure}[H]
	\begin{center}
		\caption{From \eqref{eq:dot_theta} the solution circles clockwise.}
		\includegraphics[width=0.65\textwidth]{wk2/floquet.png}
	\end{center}
\end{figure}
From \eqref{eq:dot_r0}
\begin{align*}
	\dot{r} =&	\cos{\theta} \dot{q} + \sin{\theta} \dot{p}\\
	=& 	\cos{\theta} \big[p + q\big(1-q^2-p^2\big)\big] + \sin{\theta} \big[-q + p\big(1-q^2-p^2\big)\big]\\
	=& 	\cos{\theta} \big[r \cancel{\sin{\theta}} + r \cos{\theta}\big(1-r^2\big)\big] + \sin{\theta} \big[\cancel{-r \cos{\theta}} + r \sin{\theta}\big(1-r^2\big)\big]\\
	=& r \big(1-r^2\big) \numberthis \label{eq:dot_r}
\end{align*}

Substitute $r=1+\delta r$ in \eqref{eq:dot_r}
\begin{align*}
	\dot{\delta} =& \big(1+\delta\big)\big(1 -1 -2 \delta - \delta^2\big)\\
	=& - \big(1+\delta\big) \delta (2 + \delta)\\
	\approxeq & - 2 \delta \text{, so}\\
	\delta(t) \propto & e^{-2 t}
\end{align*}
So the contracting Floquet exponent is $-2$.

\section{Homework 5}
\begin{thm}[Equivariance] 
	The equivariance condition \cite[(12.1)]{ChaosBook} gives rise to the following condition on the stability matrix.
	\begin{align*}
		g A(\vec{v}) g^{-1} =& A(g \vec{x}) \text{, \cite[(12.25)]{ChaosBook}}
	\end{align*}
\end{thm}

\begin{proof}
	\begin{align*}
		g v(\vec{x})=& v{g(\vec{x})} \text{, from \cite[(12.1)]{ChaosBook}. Differentiating}\\
		\partial_k g_{ij} v_j(\vec{x})=& \partial_k v_i (g_{1j} x_j, g_{2j} x_j,... ) \text{, upon applying the summation convention}
	\end{align*}
	So the left hand side is:
	\begin{align*}
		g_{ij} \partial_k v_j(\vec{x})=& g_{ij}A_{jk}\\
		=& (gA)_{ik}
	\end{align*}
	The right hand side is:
	\begin{align*}
		\partial_k v_i (g_{1j} x_j, g_{2j{^\prime}} x_{j{^\prime}},... ) =& A_{il} g_{lk}\\
		=&(Ag)_{ik}
	\end{align*}
	Equating the left and right hand sides:
	\begin{align*}
		\big(gA(\vec{x})\big)_{ik} =& \big(A(g \vec{x})g\big)_{ik}\\
		gA(\vec{x}) =& A(g \vec{x})g\\
		gA(\vec{x})g^{-1} =& A(g \vec{x})
	\end{align*}
\end{proof}

The "two modes" equations from \cite[12.4.2]{ChaosBook} are:
\begin{align*}
	\dot{z_1} =& \mu_1 z_1 - z_1 \vert z_1 \vert^2 + c_1 \bar{z_1} z_2  \numberthis \label{eq:dot_z1}\\
	\dot{z_2} =& (1-i)z_2 + a_2z_2 \vert z_1 \vert^2+z_1^2  \numberthis \label{eq:dot_z2}
\end{align*}


\begin{thm}[Invariance of the two modes system]
	Equations \eqref{eq:dot_z1} and \eqref{eq:dot_z2} are invariant under $z_1 \rightarrow e^{i\theta} z_1,\; z_2 \rightarrow e^{2i\theta} z_2$.
\end{thm}
\begin{proof}
	If we apply the transformation, the equations become:
	\begin{align*}
		e^{i\theta} \dot{z_1} =& e^{i\theta} \mu_1 z_1 - e^{i\theta} z_1 \vert z_1 \vert^2 + c_1 \cancel{e^{-i\theta}}   \bar{z_1} e^{\cancel{2}i\theta} z_2 \\
		e^{2i\theta}\dot{z_2} =& e^{2i\theta} (1-i)z_2 + e^{2i\theta} a_2z_2 \vert z_1 \vert^2+\underbrace{(e^{i\theta}z_1)^2}_\text{$= e^{2i\theta}z_1^2$}
	\end{align*}
	The phases $e^{i\phi}$ cancel, leaving \eqref{eq:dot_z1} and \eqref{eq:dot_z2}.
\end{proof}

We split the $z_j$ into real and imaginary parts. Differentiating \eqref{eq:dot_z1}
\begin{align*}
	\dot{x_1} + i\dot{y_1}=& \mu_1(x_1+i y_1) - (x_1+i y_1)  \underbrace{(x_1^2+ y_1^2)}_\text{$= r^2$, say}  + c_1 \bar{(x_1-i y_1)} (x_2+i y_2)\\
	=& \big[\mu_1 x_1 - r^2 x_1 +c_1(x_1x_2+y_1y_2)\big]+ i\big[\mu_1y_1 - r^2 y_1 +c_1(x_1y_2-x_2y_1)\big]\\
	\dot{x_1} =&\mu_1 x_1 - r^2 x_1 +c_1(x_1x_2+y_1y_2)\\
	 =&(\mu_1 - r^2 )x_1 +c_1(x_1x_2+y_1y_2)  \numberthis \label{eq:dot_x1}\\
	\dot{y_1}=& \mu_1y_1 - r^2 y_1 +c_1(x_1y_2-x_2y_1) \\
	=& (\mu_1 - r^2) y_1 +c_1(x_1y_2-x_2y_1)  \numberthis \label{eq:dot_y1}
\end{align*}
Differentiating \eqref{eq:dot_z2}
\begin{align*}
	\dot{x_2} + i\dot{y_2} =& (1-i)(x_2+i y_2) + a_2(x_2+i y_2) (x_1^2+ y_1^2)+(x_1+i y_1)^2\\
	=& \big[x_2+y_2 + a_2 r^2 x_2 +x_1^2-y_1^2\big] + i\big[y_2-x_2+a_2 r^2 y_2 + 2 x_1y_1\big]\\
	\dot{x_2} =&x_2+y_2 + a_2 r^2 x_2 +x_1^2-y_1^2 \numberthis \label{eq:dot_x2}\\
	\dot{y_2}=& y_2-x_2+a_2 r^2 y_2 + 2 x_1y_1  \numberthis \label{eq:dot_y2}
\end{align*}

\begin{thm}[Exercise 13.6: Invariant subspace of the two modes system]
	If a trajectory has its initial point $(0,0,x_2,y_2)$, it remains within the subspace $(0,0,x_2^\prime,y_2^\prime)$ forever.
\end{thm}

\begin{proof}
	Substituting $(0,0,x_2,y_2)$ in \eqref{eq:dot_x1}, \eqref{eq:dot_y1}, \eqref{eq:dot_x2}, and \eqref{eq:dot_y2}, the equations simplify to.
	\begin{align*}
		\dot{x_1} =&0  \\
		\dot{y_1}=& 0 \\
		\dot{x_2} =&x_2+y_2  \\
		\dot{y_2}=& y_2-x_2
	\end{align*}
The results follows from \cite[Theorem 3]{hurewicz1958lectures}.
\end{proof}

\begin{align*}
	\vec{x}=& \begin{bmatrix}
		x_1&y_1&x_2&y_2
	\end{bmatrix}^\top\\
	A_{ij} =&\frac{\partial \dot{x}_i}{\partial x_j}
\end{align*}

Taking partial derivatives of \eqref{eq:dot_x1}, \eqref{eq:dot_y1}, \eqref{eq:dot_x2}, and \eqref{eq:dot_y2}
\begin{align*}
	A =& \begin{bmatrix}
		\frac{\partial \dot{x_1}}{\partial {x_1}}&\frac{\partial \dot{x_1}}{\partial {y_1}}&\frac{\partial \dot{x_1}}{\partial {x_2}}&\frac{\partial \dot{x_1}}{\partial {y_2}}\\
		\frac{\partial \dot{y_1}}{\partial {x_1}}&\frac{\partial \dot{y_1}}{\partial {y_1}}&\frac{\partial \dot{y_1}}{\partial {x_2}}&\frac{\partial \dot{y_1}}{\partial {y_2}}\\
		\frac{\partial \dot{x_2}}{\partial {x_1}}&\frac{\partial \dot{x_2}}{\partial {y_1}}&\frac{\partial \dot{x_2}}{\partial {x_2}}&\frac{\partial \dot{x_2}}{\partial {y_2}}\\
		\frac{\partial \dot{y_2}}{\partial {x_1}}&\frac{\partial \dot{y_2}}{\partial {y_1}}&\frac{\partial \dot{y_2}}{\partial {x_2}}&\frac{\partial \dot{y_2}}{\partial {y_2}}
	\end{bmatrix}\\
	=& \begin{bmatrix}
		-2x_1x_1 + \mu_1-r^2+x_2&-2y_1x_1+c_1y_2&c_1x_1&c_1y_1\\
		-2 x_1y_1 +c_1 y_2&-2y_1y_1+\mu_1-r^2-c_1x_2&-c_1y_1&c_1x_1\\
		2x_1+2a_2x_2x_1&-2y_1+2a_2x_2y_1&1+a_2r^2&1\\
		2y_1+2a_2 y_2 x_1&2x_1+2a_2y_2y_1&-1&1+a_2 r^2
	\end{bmatrix}\\
 =& \begin{bmatrix}
		\mu_1-3x_1^2+c_1x_2-y_1^2&c_1y_2-2x_1y_1&c_1x_1&c_1y_1\\
		c_1y_2-2x_1y_1&\mu_1-x_1^2-c_1x_2-3y_1^2&-c_1y_1&c_1x_1\\
		2x_1+2a_2x_1x_2&2a_2x_2y_1-2y_1&1+a_2(x_1^2+y_1^2)&1\\
		2y_1+2a_2x_1y_2&2x_1+2a_2y_1y_2&-1&1+a_2(x_1^2+ y_1^2)
	\end{bmatrix}\numberthis \label{eq:stability}
\end{align*}
\begin{thm}[G-Equivariance]
	The flow described by \eqref{eq:dot_x1}, \eqref{eq:dot_y1}, \eqref{eq:dot_x2}, and \eqref{eq:dot_y2} is equivariant under the group generated by:
	\begin{align*}
		T =& \begin{pmatrix}
			0&-1&0&0\\
			1&0&0&0\\
			0&0&0&-2\\
			0&0&2&0
		\end{pmatrix} \numberthis \label{eq:two_modes:generator}
	\end{align*}
\end{thm}

\begin{proof}
	From \cite[12.15]{ChaosBook}, the flow will be equivariant iff 
	\begin{align*}
		t(\vec{v}) =& At(\vec{x}) \numberthis \label{eq:equivariance}
	\end{align*}
	Now, from \eqref{eq:equivariance}
	\begin{align*}
		t(\vec{x})=&T\vec{x}\\
		 =&\begin{bmatrix}
			-y_1&x_1&-2y_2&2x_2  \numberthis \label{eq:equivariance_tx}
		\end{bmatrix}^\top\\
		t(\vec{v})=&T\vec{v}\\
		=&\begin{bmatrix}
			-\dot{y_1}&\dot{x_1}&-2\dot{y_2}&2\dot{x_2}  \numberthis \label{eq:equivariance:tv}
		\end{bmatrix}^\top
	\end{align*}
	We calculate the four components of \eqref{eq:equivariance}, using \eqref{eq:stability} and \eqref{eq:equivariance_tx}.
	\begin{align*}
		[A(\vec{x}) t(\vec{x})]_1 =& 
		(\mu_1-3x_1^2+c_1x_2-y_1^2)(-y_1)+(c_1y_2-2x_1y_1)x_1+(c_1x_1)(-2y_2)+c_1y_12x_2\\
		=&-\mu_1 y_1 + \cancel{3} x_1^2 y_1 -\bcancel{ c_1 x_2 y_1} + y_1^3 + \xcancel{c_1 x_1 y_2} -\cancel{2 x_1^2 y_1} - \xcancel{2} c_1 x_1 y_2 + \bcancel{2} c_1y_1x_2\\
		=& -\mu y_1 + (x_1^2 + y_1^2) y_1 + c_1(y_1x_2-x_1y_2)\\
		=& -\dot{y_1} \text{ from \eqref{eq:dot_y1}}
	\end{align*}
	
	\begin{align*}
		[A(\vec{x}) t(\vec{x})]_2 =& 	(c_1y_2-2x_1y_1)(-y1)+(\mu_1-x_1^2-c_1x_2-3y_1^2)x_1+(-c_1y_1)(-2y_2)+(c_1x_1)(2x_2)\\
		=& \cancel{-c_1y_1y_2} + \xcancel{2x_1y_1^2} + \mu_1 x_1-x_1^3-\bcancel{c_1x_1x_2}-\xcancel{3}y_1^2x_1+\cancel{2} c_1y_1y_2 + \bcancel{2}c_1x_1x_2\\
		=& \mu_1 x_1 -r^2 x_1 + c_1(y_1y_2+x_1x_2)\\
		=& \dot{x_1}  \text{ from \eqref{eq:dot_x1}}
	\end{align*}
	\begin{align*}
		[A(\vec{x}) t(\vec{x})]_3 =& (2x_1+2a_2x_1x_2)(-y_1) + (2a_2x_2y_1-2y_1) x_1 + [1+a_2(x_1^2+y_1^2)](-2y_2) + 2 x_2\\
		=& -2 x_1 y_1 -2 a_2 x_1 x_2 y_1 + 2 a_2 x_2 y_1 x_1 - 2y_1 x_1 - 2 y_2 -2 a_2 r^2 y_2 + 2x_2\\
		=& -2 \big[ x_1 y_1 + \cancel{a_2 x_1 x_2 y_1} - \cancel{a_2 x_2 y_1 x_1} + y_1 x_1 + y_2 +  a_2 r^2 y_2 - x_2\big]\\
		=& -2 \dot{y_2}  \text{ from \eqref{eq:dot_y2}}
	\end{align*}
	
	\begin{align*}
		[A(\vec{x}) t(\vec{x})]_4 =& (2y_1+2a_2x_1y_2)(-y_1) + (2x_1+2a_2y_1y_2)x_1 - (-2 y_2) + [1+a_2(x_1^2+ y_1^2)](2x_2)\\
		=&2\big[(y_1+a_2x_1y_2)(-y_1) + (x_1+a_2y_1y_2)x_1 + y_2 + x_2+a_2(x_1^2+ y_1^2)x_2\big]\\
		=&2\big[-y_1^2 - \cancel{a_2 x_1 y_1 y_2} +x_1^2 +  \cancel{a_2 x_1 y_1 y_2} + y_2 + x_2+ a_2(x_1^2+ y_1^2)x_2\big]\\
		=& 2 \dot{x_2}  \text{ from \eqref{eq:dot_x2}}
	\end{align*}
	Comparing the forgoing expressions with \eqref{eq:equivariance:tv}, we see that \eqref{eq:equivariance} is satisfied.
\end{proof}

\begin{thm}[Chaosbook 13.37]
	For the two modes system, if we use the template point
	\begin{align*}
		\hat{x}^\prime =& (1,0,0,0) \text{, the phase velocity is:}\\
		\dot{\phi}(\hat{x}) =& \frac{v_2(\hat{x})}{\hat{x}_1} \numberthis \label{eq:two:modes:phase}
	\end{align*}
\end{thm}
\begin{proof}
	From \cite[(13.8)]{ChaosBook}
	\begin{align*}
		\dot{\phi}(\hat{x}) =& \frac{\braket{v(\hat{x})\vert t^\prime}}{\braket{t(\hat{x})\vert t^\prime}}\numberthis \label{eq:two:modes:phase1}\\
		t^\prime =& T \hat{x}^\prime\\
		=& \begin{pmatrix}
			0&-1&0&0\\
			1&0&0&0\\
			0&0&0&-2\\
			0&0&2&0
		\end{pmatrix} \begin{pmatrix}
			1\\
			0\\
			0\\
			0
		\end{pmatrix}\text{, from \eqref{eq:two_modes:generator}}\\
		=& \begin{pmatrix}
				0\\
				1\\
				0\\
				0
		\end{pmatrix}
	\end{align*}
	\begin{align*}
		v(\hat{x}^\prime) =& (v_1, v_2, v_3, v_4)\\
		\braket{v(\hat{x})\vert t^\prime} =& v_2\\
		t(\hat{x}) =&  T \begin{pmatrix}
								\hat{x}_1\\
								\hat{y}_1\\
								\hat{x}_2\\
								\hat{y}_2
							\end{pmatrix}\\
			=& \begin{pmatrix}
				- \hat{y}_1\\
				\hat{x}_1\\
				-2 \hat{y}_2\\
				2\hat{x}_2
			\end{pmatrix}\\
		\braket{t(\hat{x})\vert t^\prime}=& \hat{x}_1
	\end{align*}
Equation \eqref{eq:two:modes:phase} follows on substituting the expressions for the numerator and denominator in \eqref{eq:two:modes:phase1}.
\end{proof}
\section{Homework 6}

\begin{align*}
	x_0 =& (0.d_1 d_2 d_3 ...)_2\\
	f(x_0) =& 2 \times \begin{cases}
		(0.0 d_2 d_3 d_4...)_2 & \text{for } d_1 = 0\\
		[1- (0.1 d_2 d_3 d_4...)_2] & \text{for } d_1 = 1
	\end{cases}\\
		=& \begin{cases}
			(0.d_2 d_3 d_4...)_2 & \text{for } d_1 = 0\\
			[0.(1-d_2) (1-d_3) (1-d_4)...]_2 & \text{for } d_1 = 1
		\end{cases}\\
	=& (0.d^\prime_2 d^\prime_3 d^\prime_4 ...)_2
\end{align*}
\section{Homework 7}
\subsection{H\'enon mapping}
\cite{henon1976two}
\begin{align*}
	(x_{n+1},y_{n+1}) =& (1-ax_n^2+by_n,x_n) \numberthis \label{eq:henon:recurrence}\\
	x_n =& y_{n+1}\\
	1-ax_n^2+by_n =& x_{n+1}\\
	y_n =& \frac{x_{n+1} + a  y_{n+1}^2-1}{b}
\end{align*}
At a fixed point, \eqref{eq:henon:recurrence} becomes
\begin{align*}
	(x,y) =& (1-ax^2+by,x)\text{, ie}\\
	x =& y \text{, and}\\
	x =& 1-ax^2 +by\\
	=& 1-ax^2 +bx\text{. Gathering terms}\\
	ax^2 + (1-b)x = &1\\
	a\big[x^2 + 2\frac{1-b}{2a}x + \big(\frac{1-b}{2a}\big)^2\big]  = &1+ \frac{(1-b)^2}{2a}\\
	a\big[x+ \frac{1-b}{2a}\big]^2= &1+ \frac{(1-b)^2}{4a}\\
	\big[x+ \frac{1-b}{2a}\big]^2= &\frac{1+ \frac{(1-b)^2}{4a}}{a}\\
	x = - \frac{1-b}{2a} \pm \sqrt{\frac{1+ \frac{(1-b)^2}{4a}}{a}}
\end{align*}
\bibliographystyle{unsrt}
\addcontentsline{toc}{section}{Bibliography}
\bibliography{../dynamics}

\end{document}
